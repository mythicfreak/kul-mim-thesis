\chapter{Conclusion}\label{chap:conclusions}
The centerpiece of this thesis is radical innovation. Such innovations have
regularly been seen in the past, and seem to be a key factor in the long term
growth of firms or even entire regions. It follows that identifying such radical
innovations early is critical. Chapter \ref{chap:intro} introduced the reader to
a framework based on three dimensions that does just this. These three
dimensions are novelty in knowledge origins, novelty in functionality
and technological impact. Innovations that score high on all three components
are very likely to be radical. In addition, the framework proposes patent
indicators to automatically score innovations using complex computer algorithms.

In this text however, we restricted ourselves to a thorough manual assessment of
a few innovations in the field of diagnostic medical imaging. To that end, we
presented a short introduction to this field in \autoref{chap:imaging}. In
particular, we discussed the history, technical background, recent advancements
and future expectations of four imaging modalities: radiography, computer
tomography, magnetic resonance imaging and nuclear medicine imaging. We also
briefly discussed computer aided detection and diagnosis.

Once we had a basic understanding of the field, \autoref{chap:inventions}
provided us with a deeper understanding of some more recent innovations in the
field. For each innovation, scores were given based on the assessment sheet in
the appendix. These innovations include digital radiography, electron beam
computed tomography, magnetic resonance imaging, fluorodeoxyglucose tracers and
modern computer aided detection and diagnosis techniques. The scores turned out
lower than expected, and we listed a few possible reasons for this.

This work can serve as a basis to validate and further refine the radical
innovation framework. By comparing the results of this manual assessment with
the outcome of an automatic assessment based on patent indicators, potential
discrepancies and flaws in the framework can hopefully be found and alleviated.