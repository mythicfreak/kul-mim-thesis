\section{PET invention: FDG tracers}
In this assessment related to PET we focus on something slighly unusual. Not a
component or new technique used in the scanner, but on one of the popular
tracers used during such examinations.

\subsection{Defining the technology}
PET scanners work by detecting positrons formed by the decay of radioactive
tracers, as explained in \autoref{chap:nuclearimg}. The most commonly used
isotope is fluorine-18. It is very suited for PET because it almost exclusively
emits positrons when decaying to oxygen. Its half life of 110 minutes gives it
another attractive property.

However, the fluorine isotope is only responsible for the emission of positrons,
it cannot be absorbed by or transported in the human body on its own. That is
where 2-deoxy-2-fluoro-D-glucose or, more conveniently fluorodeoxyglucose (FDG),
comes into play. One of its hydroxyl (-OH) groups can be replaced by
fluorine-18. Because it is analogous to normal glucose, its uptake in the body
is also similar. However, unlike normal glucose it cannot be completely
metabolised as long as the fluorine isotope is present. Instead, it gets stuck
in the cells until the fluorine decays into oxygen and it can form a normal
hydroxyl group again.

When injected into the body, a PET scanner can visualize the distribution of the
tracer throughout the various tissues. This way, phyicians can monitor glucose
metabolism and spot abnormalities or deviations. Those can be caused by tumors
for example, which typically grow very fast and consequently need lots of
energy. Brain activity can also be measured regionally because active parts of
the brain tend to metabolise more glucose.

%impact: neuroscience

\subsection{Assessing novelty in functionality}

\subsubsection{Novelty in components}

\subsubsection{Novelty in natural effects exploited}

\subsubsection{Scores}
In \autoref{tbl:funcscores4} we give an overview of the scores on the various
topics.

\begin{table}[h]
\centering
\begin{tabular}{l l}
\hline
\multicolumn{2}{|c|}{Novelty in functionality} \\
\hline
A. Novelty of components & B. Novelty in natural effects exploited\\
A1) & B1)\\ 
A2) & B2)\\ 
\hline
\end{tabular}
\caption{Novelty in functionality scores}
\label{tbl:funcscores4}
\end{table}

\subsection{Assessing novelty in knowledge origins}
\subsubsection{Novelty of scientific origins}

\subsubsection{Novelty of technological origins}

\subsubsection{Scores}
\begin{table}[h]
\centering
\begin{tabular}{l l}
\hline
\multicolumn{2}{|c|}{Novelty in knowledge origins} \\
\hline
A. Novelty of scientific origins & B. Novelty of technological origins\\
A1) & B1)\\ 
A2) & B2)\\ 
\hline
\end{tabular}
\caption{Novelty in knowledge origins scores}
\label{tbl:origscores4}
\end{table}

\subsection{Assessing technological impact}
\subsubsection{Performance increase}

\subsubsection{Technological accumulation}
\paragraph{Broadness of impact}

\paragraph{Magnitude of impact}

\paragraph{Novelty of impact}

\subsubsection{Obsoleting previous technologies}

\subsubsection{Scores}
\begin{table}[h]
\centering
\begin{tabular}{l l l}
\hline
\multicolumn{3}{|c|}{Technological impact} \\
\hline
A. Performance increase & B. Tech. accumulation & Obsoleting previous tech.\\
A)   & B1 a)  --- b)  & C) \\ 
     & B2 a)  --- b)  & \\
     & B3 a)  --- b)  & \\
\hline
\end{tabular}
\caption{Technological impact scores}
\label{tbl:impactscores4}
\end{table}