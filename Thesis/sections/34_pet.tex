\section{Nuclear medicine invention: F-FDG tracers}
In this assessment related to nuclear medicine we focus on something slighly
unusual. Not a component or new technique used in the scanner, but on one of the
popular tracers used during such examinations.

\subsection{Defining the technology}
PET scanners work by detecting positrons formed by the decay of radioactive
tracers, as explained in \autoref{chap:nuclearimg}. The most commonly used
isotope is fluorine-18. It is very suited for PET because it almost exclusively
emits positrons when decaying to oxygen-18. Its half life of 110 minutes gives
it another attractive property.

However, the fluorine isotope is only responsible for the emission of positrons,
it cannot be absorbed by or transported in the human body on its own. That is
where 2-deoxy-2-fluoro-D-glucose or, more conveniently fluorodeoxyglucose (FDG),
comes into play. One of its hydroxyl (-OH) groups can be replaced by
fluorine-18. Because it is analogous to normal glucose, its uptake in the body
is also similar. However, unlike normal glucose it cannot be completely
metabolised as long as the fluorine isotope is present. Instead, it gets stuck
in the cells until the fluorine decays into oxygen and it can form a normal
hydroxyl group again.

When injected into the body, a PET scanner can visualize the distribution of the
tracer throughout the various tissues. This way, phyicians can monitor glucose
metabolism and spot abnormalities or deviations. Those can be caused by tumors
for example, which typically grow very fast and consequently need lots of
energy. Brain activity can also be measured regionally because active parts of
the brain tend to metabolise more glucose.

Synthesis of FDG is possible in a number of ways, but most are rather
sophisticated. 

%impact: neuroscience, oncology

%1948: NO, Fick principle, cerebral bloodflow (global)
%1977: Sokoloff, C14-glucose [20]
%1979: Reivich [21], FDG, Ido et al. [22]
%uses: table 1

\subsection{Assessing novelty in functionality}

\subsubsection{Novelty in components}
The F-FDG tracer is a composite made out of both FDG and fluorine-18. Earlier
tracers also used FDG, but the isotopes were different. Carbon-14 was a popular
choice \cite{radiopharma}. The fluorine-18 seems to be exclusively used as a
radioactive tracer. It is sometimes also combined with other molecules, for
example to track dopamine instead of glucose \cite{fluorodopa, fdopamine}.
Another isotope (fluorine-19) is used as a contrast agent in MRI studies.

The first paper describing the synthesis of F-FDG already mentioned specific
usage as a radiopharmaceutical \cite{firstF-FDG}. To our knowledge, it has not
been used outside this context. Within this context however, it was used towards
many different goals in biomedical research such as neuroscience and oncology. 

\subsubsection{Novelty in natural effects exploited}
The tracer isotope exploits radioactive decay, more specifically positron
emission decay. Considering PET works exclusively with positron emission decay,
this is no special feat. However, what makes fluorine-18 special is that it
hardly has any other decay mechanisms and that its half life has the ideal
length for PET studies.

The second natural effect exploited is how the human body treats FDG very
similar to normal glucose.

\subsubsection{Scores}
In \autoref{tbl:funcscores4} we give an overview of the scores on the various
topics.

\begin{table}[h]
\centering
\begin{tabular}{l l}
\hline
\multicolumn{2}{|c|}{Novelty in functionality} \\
\hline
A. Novelty of components & B. Novelty in natural effects exploited\\
A1) 6 & B1) 3\\ 
A2) 2 & B2) 1\\ 
\hline
\end{tabular}
\caption{Novelty in functionality scores}
\label{tbl:funcscores4}
\end{table}

\subsection{Assessing novelty in knowledge origins}
To make F-FDG work, a number of problems had to be overcome.

Problem 1: find a tracer that emits positrons

Solution 1: the following isotopes are known to emit positrons:  carbon-11,
nitrogen-13, oxygen-15, fluorine-18, sodium-22, aluminium-26, potassium-40, 
and iodine-121

KO1: nuclear physics

Problem 1.1: find a tracer with a suitable half life

Solution 1.1: 

KO2: nuclear physics

Problem 2: find a suitable transport molecule that mimics glucose

Solution 2: replace one hydroxyl group in normal glucose with a tracer 

KO3: biochemistry


\subsubsection{Scores}
\autoref{tbl:origscores4} shows the results. Note that there are no technical
knowledge origins. The scientific knowledge origins are rather straightforward
for these kinds of applications.

\begin{table}[h]
\centering
\begin{tabular}{l l}
\hline
\multicolumn{2}{|c|}{Novelty in knowledge origins} \\
\hline
A. Novelty of scientific origins & B. Novelty of technological origins\\
A1) & B1) N/A\\ 
A2) & B2) N/A\\ 
\hline
\end{tabular}
\caption{Novelty in knowledge origins scores}
\label{tbl:origscores4}
\end{table}

\subsection{Assessing technological impact}
\subsubsection{Performance increase}

\subsubsection{Technological accumulation}
\paragraph{Broadness of impact}

\paragraph{Magnitude of impact}

\paragraph{Novelty of impact}

\subsubsection{Obsoleting previous technologies}

\subsubsection{Scores}
\begin{table}[h]
\centering
\begin{tabular}{l l l}
\hline
\multicolumn{3}{|c|}{Technological impact} \\
\hline
A. Performance increase & B. Tech. accumulation & Obsoleting previous tech.\\
A)   & B1 a)  --- b)  & C) \\ 
     & B2 a)  --- b)  & \\
     & B3 a)  --- b)  & \\
\hline
\end{tabular}
\caption{Technological impact scores}
\label{tbl:impactscores4}
\end{table}