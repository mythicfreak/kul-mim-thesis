\section{Computer-aided Detection and Diagnosis}
This section on Computer-aided Detection and Diagnosis (CAD, not to be
confused with Computer-aided Design) deals not with the image creation process,
but with the subsequent processing. In this sense it is the odd one out.

CAD is one of the major research areas in diagnostic medical imaging
\cite{cadhistory, suetens}. CAD technologies help radiologists with the actual
detection and diagnosis of pathologies in the image. Sometimes a further
distinction is made between detection (CADe) and diagnosis (CADx). Either way,
CAD does not attempt to replace physicians. You can think of it more as a second
opinion. This is in contrast with Automated Computer Diagnosis (ACD) where the
computer (supposedly) does all the work and no physician is needed anymore.

The rise of CAD is tied to the increased workload of radiologists. Decades ago,
each radiological examination contained just a few images. Today, with
multi-slice CT scans and full body MRI scans, a single examination can contain
thousands of 2D images. Clearly, some form of automation would be more than
welcome to reduce the radiologists' workload.

A (non-exhaustive) list of possible application of CAD include detection of
vertebral fractures on radiographs, detection of microcalcifications on
mammograms, detection of cranial aneurysms in MRA images and detection of lung
nodules in CT scans \cite{cadhistory}.

Note: in the following sections related to CAD, we assume that the reader is
familiar with the basics of statistical binary classification theory. The most
important concepts are True Positives (TP), False Positives (FP), True Negatives
(TN), False Negatives (FN), accuracy, sensitivity, specificity, and Receiver
Operating Characteristic (ROC) curves.

%US4907156 \cite{chan1990method}

\subsection{History}
Early in the 1980's, when large scale research on CAD began, the most common
techniques fell under basic image processing. This is a branch of signal
processing, applied to 2D signals (i.e., images). In the following years, a new
field called Computer Vision formed around the subject. 

To get a feel for such image processing techniques, consider a special filter
to accentuate lung nodules, and another one to suppress them. Next, apply these
two filters to the same image, and subtract the suppressed from the accentuated
version. The result should then be a completely black image, save for the
nodule-like structures in the image. 

Early on, most research focused on ACD. It was not a success at first,
because computers were not yet powerful enough, and advanced image processing
algorithms were not yet available. The researchers had high hopes for the
future, but unfortunately ACD never caught on. The algorithms' sensitivity and
specificity simply did not match up to that of real physicians. That is why most
researchers soon made the switch to CAD. To measure the performance of CAD
systems, we do not only look at the outcome of the algorithm, but how this outcome
combined with the physician's expertise can create synergy. Even if the
sensitivity and specificity of the CAD algorithm are lower than that of the
average physician, as long as the synergy is big enough, their combined use can
trump that of the physician alone. This was proven for the first time in
\cite{cadsynergy}.

At the time, the most popular topics were related to cardiovascular diseases,
lung cancer and breast cancer because of their high impact. This trend has
largely continued, and even in recent year researchers still focus mostly on the
latter two topics. \cite{cadhistory}

The first scanner with built-in CAD detection was approved by the U.S. Food and
Drug Administration (FDA) in 1998.

\subsection{Technical background}\label{ssec:cadtech}
Contrary to the discussion of the previous imaging modalities, we will not go
into the technical details of these techniques. There reason is simple: there
are too many of them, and they all work in a very different way. The interested
reader is invited to read a book on the subject such as \cite{imgprocbook}.

Fortunately, most CAD systems follow a generic scheme, independent of the
underlying technology \cite{lungcadreview}. First of all, the organ of interest
(e.g., the lungs) is segmented (I), and the rest of the image is discarded.
Next, the object to be detected is somehow enhanced (II). The subtraction
technique outlined above could work here. Then, these candidate objects have to
be detected and segmented (III). After that, some kind of feature analysis is
performed and the candidates are properly classified (IV). Most techniques will
still have plenty of false positives at this point, so FP reduction is required
(V). If simple detection was the goal, the remaining candidates are the end
result of this scheme. Else, an extra diagnosis (characterization) step is
required (VI).

\subsection{Recent advancements}\label{ssec:cadadv}
In the last two decades, more and more interest has been generated about the
implementation of artificial intelligence (AI) and machine learning techniques
in CAD. Examples of such techniques include Discriminant Analysis (DA),
Artificial Neural Networks (ANN), Support Vector Machines (SVM) and decision
tree models such as Random Forests (RF) \cite{lungcadreview, ginneken}. The
exact details of each method are out of scope for this text, but fortunately
most can be treated as black boxes with highly similar in- and outputs. 

Machine learning always works in two phases. The learning phase and the
prediction phase. During the learning phase, a large number of class-annotated
sample images are provided from which the algorithm can infer the differences
between various classes. Examples of simple binary classes are tumors vs.
non-tumors or malignant tumors vs. benign tumors. Once training has completed,
an unannotated picture can be provided in the prediction phase. The algorithm
will then predict - based on its internal model - to which class it belongs.

Of course it is possible to use a complete image, pixel by pixel, as input.
However, this is computationally very expensive. Instead, features are typically
calculated to describe a certain region of interest in an image. Simple
features are the minimum, average and maximum intensity, but they can become
arbitrarily complex.

\subsection{Future expectations}
\cite{cadhistory} sees future potential of CAD in combination with Picture
Archiving and Communication Systems (PACS). Since the digital revolution,
hospitals had to come up with a new way of storing and managing all this imaging
data. PACS was the solution, and today every modern hospital has a PACS in one
form or another. This means every hospital sits on piles of old and unused
medical images. By combining CAD, AI techniques such as data mining and PACS,
research could gain a serious momentum boost and unforeseen applications could
pop up. Furthermore, integrating CAD as a PACS module would make it more
accessible and promote adoption.

Aside from that, researchers will keep working on the underlying algorithms.
This will hopefully bring steady improvements in performance.
