\chapter{Overview of radical inventions}\label{chap:inventions}

\section{Introduction}
In this chapter, we perform the assessment of five inventions based on the
framework described in \autoref{chap:intro}. Chapter \ref{chap:imaging}
already described various breakthroughs in diagnostic medical imaging, but most of them
predate 1980. Fortunately researchers in the field have come up with plenty of
other innovations in the last thirty years, some of which will be detailed
below. We included one invention per imaging modality described, plus one more
related to computer aided detection and diagnosis (CAD).

For each innovation, a technological definition is provided wherein the purpose
or goal is outlined, and the components plus their interactions are explained
based on a thorough literature review. Next, the innovation is scored based on
the assessment sheet in \autoref{app:score}.

\section{Radiography invention: flat panel matrix detector}
\subsection{Defining the technology}

\subsection{Assessing novelty in functionality}

\subsection{Assessing novelty in origins}

\subsection{Assessing technological impact}

\section{CT invention: Electron-beam CT}
Our next assessment deals with an innovation related to computed tomography. In
particular, we take a closer look at Electron-beam Computed Tomography (EBCT),
also know as Ultrafast CT.

\subsection{Defining the technology}
Traditional CT scanners have an X-ray tube embedded in the toroid body. By
mechanically rotating the tube along with the detector, projections from an
arbitrary angle can be captured.

EBCT also needs to be able to capture projections from any angle, but takes
another route. Remember that in a regular X-ray tube current flowing through the
cathode releases electrons. The electrons are accelerated towards the anode by
applying a voltage across the tube. When the electrons hit the anode at high
speed, they release part of their energy as X-rays. The same principle is used
in EBCT, except that the tube is physically split up in two dedicated parts. One
is the cathode or electron gun andis placed along the patient's longitudinal
axis. The other is the anode and is shaped in a ring around the patient. Using
magnetic fields, the electrons fired from the cathode are deflected onto this
ring, where they produce X-rays that can be captured by the detector as usual
\cite{suetens}. \autoref{fig:ebctscanner} illustrates this.

\begin{figure}[ht]
\begin{center}
  \includegraphics[width=\linewidth]{img/EBCT.png}
  \caption{Rendering of an EBCT scanner's inner workings \cite{suetens}.}
  \label{fig:ebctscanner}
\end{center}
\end{figure}

The main advantage of this setup is that the X-ray source rotation is no longer
mechanical. This allows for faster sweeps, which in turn make it easier to image
moving structures such as the heart. One very prominent application is the
detection of calcifications from atherosclerosis in the coronal arteries
\cite{ultrafastcad}.

More recently, these EBCT scanners have received strong competition from
multislice helical CT scanners. The latter enjoy a widespread adoption and are
also less constly. However, their rotation speed still cannot match that of
Ultrafast CT \cite{ultravshelical}. In addition, this techniques requires a
lower dose compared to traditional CT, and much lower than a CT angiography
\cite{ultralowdose}.

\subsection{Assessing novelty in functionality}

\subsubsection{Novelty in components}
As outlined above, the components are largely the same as in conventional CT
scanners. Only their location is different. One additional component is the coil
to generate the deflecting magnetic field. Such electromagnets were used before
in imaging, particularly in MRI, but for a very different purpose.

\subsubsection{Novelty in natural effects exploited}
The same story applies for novelty in natural effects exploited: few new effects
except bending of the electron beam using magnetic fields. The same principle
was used much earlier in Cathode Ray Tube (CRT) monitors and televisions.

\subsubsection{Scores}
In \autoref{tbl:funcscores2} we give an overview of the scores on the various
topics.

\begin{table}[h]
\centering
\begin{tabular}{l l}
\hline
\multicolumn{2}{|c|}{Novelty in functionality} \\
\hline
A. Novelty of components & B. Novelty in natural effects exploited\\
A1) 3 & B1) 3\\ 
A2) 5 & B2) 5\\ 
\hline
\end{tabular}
\caption{Novelty in functionality scores}
\label{tbl:funcscores2}
\end{table}

\subsection{Assessing novelty in knowledge origins}
Again, we look into the problems that had to be solved to find the knowledge
origins. We only look at the problems that are different in comparison to a
traditional CT scanner.

Problem 1: deflect the electron beam onto the anode ring.

Solution 1: use magnetic fields (electric field deflection only works for small angles)

KO1: electromagnetism, CRT monitors

\subsubsection{Scores}
\begin{table}[h]
\centering
\begin{tabular}{l l}
\hline
\multicolumn{2}{|c|}{Novelty in knowledge origins} \\
\hline
A. Novelty of scientific origins & B. Novelty of technological origins\\
A1) & B1)\\ 
A2) & B2)\\ 
\hline
\end{tabular}
\caption{Novelty in knowledge origins scores}
\label{tbl:origscores2}
\end{table}

\subsection{Assessing technological impact}
\subsubsection{Performance increase}

\subsubsection{Technological accumulation}
\paragraph{Broadness of impact}

\paragraph{Magnitude of impact}

\paragraph{Novelty of impact}

\subsubsection{Obsoleting previous technologies}

\subsubsection{Scores}
\begin{table}[h]
\centering
\begin{tabular}{l l l}
\hline
\multicolumn{3}{|c|}{Technological impact} \\
\hline
A. Performance increase & B. Tech. accumulation & Obsoleting previous tech.\\
A)   & B1 a)  --- b)  & C) \\ 
     & B2 a)  --- b)  & \\
     & B3 a)  --- b)  & \\
\hline
\end{tabular}
\caption{Technological impact scores}
\label{tbl:impactscores2}
\end{table}

\section{MRI invention}
Normally we only consider inventions from 1980 onwards, but for this section we
would like to make an exception. The invention of the MRI scanner itself was a
veritable breakthrough, and still happened relatively close to 1980. Because we
already had a close look at MRI in \autoref{sec:mri}, we will skip the
technology definition and go straight to the assessment.

\subsection{Assessing novelty in functionality}
\subsubsection{Novelty in components}
The purpose of MRI scanners is largely the same as that of CT scanners: create
cross-sectional images to look into patient's bodies without using invasive
techniques. The essential components of CT are the X-ray source and the
detector. With MRI, we again have a signal source and a signal detector.
However, this time the source is a large electromagnet, and the detector is a
quadrature detector.

Electromagnets are used in various other technologies to do the same thing:
generate magnetic fields. Examples include motors, generators, transformers,
relays and loudspeakers. In that sense the choice was not very special.

Similarly, quadrature detectors are used universally whenever signal
demodulation is required \cite{quadrature}. Remember that MRI operates in the
radio frequency range, so ordinary radio technology could be repurposed for use
in these scanners.

In summary, these components were not used in medical imaging before the advent
of MRI. Yet, once it was decided to exploit radio waves and magnetic fields, the
choice of components was rather trivial.

\subsubsection{Novelty in natural effects exploited}
MRI has one thing in common with the other imaging modalities discussed: they
all use electromagnetic radiation in one form or another. Nevertheless, the way
this natural effect is exploited here is considerably different from other
modalities. In radiography and CT, we measure the X-ray attenuation when
traveling through the human body. With nuclear medicine, we essentially detect
photons emitted by radioactive materials. MRI on the other hand influences and
measures the magnetic properties of tissues, in particular the proton density,
the $T_1$ time or the $T_2$ time.

In a way, every researcher that ever worked on imaging based on magnetic fields
contributed to the eventual creating of MRI. Of course some groups took a
slightly different turn (e.g. measuring properties of electrons instead of
protons), but their larger goal was the same. To our knowledge no one
experimented with these specific magnetic properties for any other purposes.

\subsubsection{Scores}
In \autoref{tbl:funcscores3} we give an overview of the scores on the various
topics.

\begin{table}[h]
\centering
\begin{tabular}{l l}
\hline
\multicolumn{2}{|c|}{Novelty in functionality} \\
\hline
A. Novelty of components & B. Novelty in natural effects exploited\\
A1) 10 & B1) 8\\ 
A2) 3 & B2) 8\\ 
\hline
\end{tabular}
\caption{Novelty in functionality scores}
\label{tbl:funcscores3}
\end{table}

\subsection{Assessing novelty in knowledge origins}
Once again, we start with a list of problems and corresponding solutions.

Problem 1: measure proton density in tissue

Problem 1.1: align net magnetization vectors

Solution 1.1: generate a large magnetic field

KO1: electromagnetism

Problem 1.1.1: increase field strength, lower power consumption

Solution 1.1.1: superconductivity by extreme cooling

KO2: physics (cryogenics), material science

Problem 1.2: measure magnetization vector

Solution 1.2a: disturb equilibrium with the appropriate pulse sequence

KO3: theory from electromagnetism, but to our knowledge never exploited before

Solution 1.2b: record relaxation phenomena using quadrature detector

KO4: analogue electronics, radio equipment 

Problem 2: localize position using signal

Solution 2: apply a magnetic field gradient

KO5: electromagnetism, gradients recently used in \cite{nanogradients1,
nanogradients2}, older: \cite{oldgradients}.

\subsubsection{Scores}
None of these knowledge origins, except the usage of specific pulses, is out of
the ordinary. Radio equipment is also a new technological origin, but its
purpose in other inventions is very similar. That is why in
\autoref{tbl:origscores3} we give low scores except for this technological
origin.

\begin{table}[h]
\centering
\begin{tabular}{l l}
\hline
\multicolumn{2}{|c|}{Novelty in knowledge origins} \\
\hline
A. Novelty of scientific origins & B. Novelty of technological origins\\
A1) 2 & B1) 6\\ 
A2) 2 & B2) 7\\ 
\hline
\end{tabular}
\caption{Novelty in knowledge origins scores}
\label{tbl:origscores3}
\end{table}

\subsection{Assessing technological impact}
\subsubsection{Performance increase}
Although we have portrayed MRI as the successor to CT in the section above, this
is not completely true. MRI allows us to visualize things we could not before
with CT. For example, cancerous tissue is sometimes easier to spot on MRI scans.
Whereas on CT scans the attenuation properties of both tissue types might be
very similar, malignant tumors appear to exhibit a larger $T_1$ value than
surrounding tissue \cite{mrihistory}.

On the other hand, CT has the advantage when it comes to visualizing bone.
Because it contains few free hydrogen atoms, it is very difficult to spot on an
MRI scan. X-rays however are ideal for this type of diagnosis.

In conclusion, it is very difficult to unilaterally award the performance
trophy to either side. 

\subsubsection{Technological accumulation}
\paragraph{Broadness of impact}
MRI generated a lot of spin-off technologies, including Magnetic Resonance
Angiography (MRA), Magnetic Resonance Spectroscopy (MRS) and functional MRI
(fMRI). These are of course all part of the same imaging family, and we could
not find any impact outside of this family in the literature. Perhaps
mining through patents can show some interesting results. 

\paragraph{Magnitude of impact}
Although the broadness is limited, the magnitude of impact is substantial.
Especially for the direct spin-offs listed above, but also indirectly through
the spin-offs of those spin-offs.

\paragraph{Novelty of impact}
As stated above, we only identified direct impact on its immediate spin-off
technologies. The degree of novelty is thus very low.

\subsubsection{Obsoleting previous technologies}
As stated before in \autoref{ssec:mrifuture}, MRI has not replaced CT, but is
expected to gain more traction in the future. For now, the two technologies are
complimentary: each is better for a specific subcategory of diagnostic imaging. 

\subsubsection{Scores}
\autoref{tbl:impactscores3} shows the final scores.

\begin{table}[h]
\centering
\begin{tabular}{l l l}
\hline
\multicolumn{3}{|c|}{Technological impact} \\
\hline
A. Performance increase & B. Tech. accumulation & Obsoleting previous tech.\\
A) 5 & B1 a) 4 --- b) 2 & C) 3\\ 
     & B2 a) 8 --- b) 8 & \\
     & B3 a) 1 --- b) 1 & \\
\hline
\end{tabular}
\caption{Technological impact scores}
\label{tbl:impactscores3}
\end{table}

\section{PET invention: FDG tracers}
In this assessment related to PET we focus on something slighly unusual. Not a
component or new technique used in the scanner, but on one of the popular
tracers used during such examinations.

\subsection{Defining the technology}
PET scanners work by detecting positrons formed by the decay of radioactive
tracers, as explained in \autoref{chap:nuclearimg}. The most commonly used
isotope is fluorine-18. It is very suited for PET because it almost exclusively
emits positrons when decaying to oxygen. Its half life of 110 minutes gives it
another attractive property.

However, the fluorine isotope is only responsible for the emission of positrons,
it cannot be absorbed by or transported in the human body on its own. That is
where 2-deoxy-2-fluoro-D-glucose or, more conveniently fluorodeoxyglucose (FDG),
comes into play. One of its hydroxyl (-OH) groups can be replaced by
fluorine-18. Because it is analogous to normal glucose, its uptake in the body
is also similar. However, unlike normal glucose it cannot be completely
metabolised as long as the fluorine isotope is present. Instead, it gets stuck
in the cells until the fluorine decays into oxygen and it can form a normal
hydroxyl group again.

When injected into the body, a PET scanner can visualize the distribution of the
tracer throughout the various tissues. This way, phyicians can monitor glucose
metabolism and spot abnormalities or deviations. Those can be caused by tumors
for example, which typically grow very fast and consequently need lots of
energy. Brain activity can also be measured regionally because active parts of
the brain tend to metabolise more glucose.

%impact: neuroscience

\subsection{Assessing novelty in functionality}

\subsubsection{Novelty in components}

\subsubsection{Novelty in natural effects exploited}

\subsubsection{Scores}
In \autoref{tbl:funcscores4} we give an overview of the scores on the various
topics.

\begin{table}[h]
\centering
\begin{tabular}{l l}
\hline
\multicolumn{2}{|c|}{Novelty in functionality} \\
\hline
A. Novelty of components & B. Novelty in natural effects exploited\\
A1) & B1)\\ 
A2) & B2)\\ 
\hline
\end{tabular}
\caption{Novelty in functionality scores}
\label{tbl:funcscores4}
\end{table}

\subsection{Assessing novelty in knowledge origins}
\subsubsection{Novelty of scientific origins}

\subsubsection{Novelty of technological origins}

\subsubsection{Scores}
\begin{table}[h]
\centering
\begin{tabular}{l l}
\hline
\multicolumn{2}{|c|}{Novelty in knowledge origins} \\
\hline
A. Novelty of scientific origins & B. Novelty of technological origins\\
A1) & B1)\\ 
A2) & B2)\\ 
\hline
\end{tabular}
\caption{Novelty in knowledge origins scores}
\label{tbl:origscores4}
\end{table}

\subsection{Assessing technological impact}
\subsubsection{Performance increase}

\subsubsection{Technological accumulation}
\paragraph{Broadness of impact}

\paragraph{Magnitude of impact}

\paragraph{Novelty of impact}

\subsubsection{Obsoleting previous technologies}

\subsubsection{Scores}
\begin{table}[h]
\centering
\begin{tabular}{l l l}
\hline
\multicolumn{3}{|c|}{Technological impact} \\
\hline
A. Performance increase & B. Tech. accumulation & Obsoleting previous tech.\\
A)   & B1 a)  --- b)  & C) \\ 
     & B2 a)  --- b)  & \\
     & B3 a)  --- b)  & \\
\hline
\end{tabular}
\caption{Technological impact scores}
\label{tbl:impactscores4}
\end{table}

\section{CAD invention: machine learning techniques for mammography}
For this last assessment, we draw from the field of Computer-aided Diagnosis.
For the rest of this section, we will focus on the usage of machine learning
techniques for CADe in mammography. Breast cancer is one of the deadliest
cancers today, but fortunately early detection significantly improves the
chances of survival \cite{mammoairecent}.

%\cite{mammoai}

\subsection{Defining the technology}
As explained in \autoref{ssec:cadadv}, machine learning comprises a large group
of methods and techniques. We are specifically interested in (binary)
classifiers to determine whether a particular structure on a mammogram is
suspicious enough. The exact nature of this classifier - whether it is a Support
Vector Machine (SVM) or Random Forests (RF) or anything else - is of little
interest for this assessment. We will simply look at them as one group with one goal.

\subsection{Assessing novelty in functionality}

\subsubsection{Novelty in components}

\subsubsection{Novelty in natural effects exploited}

\subsubsection{Scores}
In \autoref{tbl:funcscores5} we give an overview of the scores on the various
topics.

\begin{table}[h]
\centering
\begin{tabular}{l l}
\hline
\multicolumn{2}{|c|}{Novelty in functionality} \\
\hline
A. Novelty of components & B. Novelty in natural effects exploited\\
A1) & B1)\\ 
A2) & B2)\\ 
\hline
\end{tabular}
\caption{Novelty in functionality scores}
\label{tbl:funcscores5}
\end{table}

\subsection{Assessing novelty in knowledge origins}
\subsubsection{Novelty of scientific origins}

\subsubsection{Novelty of technological origins}

\subsubsection{Scores}
\begin{table}[h]
\centering
\begin{tabular}{l l}
\hline
\multicolumn{2}{|c|}{Novelty in knowledge origins} \\
\hline
A. Novelty of scientific origins & B. Novelty of technological origins\\
A1) & B1)\\ 
A2) & B2)\\ 
\hline
\end{tabular}
\caption{Novelty in knowledge origins scores}
\label{tbl:origscores5}
\end{table}

\subsection{Assessing technological impact}
\subsubsection{Performance increase}

\subsubsection{Technological accumulation}
\paragraph{Broadness of impact}

\paragraph{Magnitude of impact}

\paragraph{Novelty of impact}

\subsubsection{Obsoleting previous technologies}

\subsubsection{Scores}
\begin{table}[h]
\centering
\begin{tabular}{l l l}
\hline
\multicolumn{3}{|c|}{Technological impact} \\
\hline
A. Performance increase & B. Tech. accumulation & Obsoleting previous tech.\\
A)   & B1 a)  --- b)  & C) \\ 
     & B2 a)  --- b)  & \\
     & B3 a)  --- b)  & \\
\hline
\end{tabular}
\caption{Technological impact scores}
\label{tbl:impactscores5}
\end{table}

\section{Conclusion}
In this chapter, we presented the results of the assessment of various
innovations related to diagnostic medical imaging. We tried to be as varied as
possible in the selection of innovations. Not only did they all stem from a
different field, but their scope and type also varied. For example, we tried a
high-level assessment of MRI as a whole, but also assessment of a whole class of
algorithms and a very specific radiopharmaceutical compound. In our newfound
experience, the assessment framework seems to work best on rather low-level, yet
tangible innovations. Nonetheless, it showed considerable flexibility when
applied to concepts it was perhaps never intended for.

Truthfully, we are slightly disappointed by the seemingly low scores of these
important innovations. We explore some reasons for this in the following
paragraphs.

We have tried to make this assessment as detailed as possible. However, medical
imaging is a very complex matter, and even our introduction in
\autoref{chap:imaging} - much like many a review paper - just barely scratch the
surface of all underlying methods and technologies. These details only become
apparent when hands-on work is performed by scientists and researchers in the
field. For example, a little known fact is that the anode in an X-ray tube gets
hot during operation, lowering performance. A simple solution is to make it spin
to cool it down. Such details will only be described in the most detailed
literature and of course in patent applications. We fear that this discrepancy
in level of detail will significantly hamper comparison with the automated
patent analysis.

The way the framework is built also seems to favor automatic patent based
analysis over manual assessment. The questions on the assessment sheet map
nicely to data that can be extracted from patents. On the other hand, manually
finding an answer to these questions without looking at patent data has proven
to be challenging at times. The granularity of the scores presents another
problem. It is very difficult to explain the difference between, for example, a
5 and a 6. In that regard we would either propose a rescaled score chart with
scores from 1 to 5. The alternative is to improve the existing score chart with a more
detailed description of how to quantify the small differences in score.

In addition, the 1980 limit made it significantly more difficult to go back to
the real breakthrough invention, the so-called seed. For example, FDG in nuclear
medicine was certainly a breakthrough but it would have been more interesting to
go back further and see who first managed to bring nuclear physics together with
biochemistry and truly invent radiopharmacology. The fact that many
imaging-related inventions spend a very long time in the research phase, taking
decades before entering the market, does not help in this regard either. On the
other hand, due to the ex-post nature of some assessment questions (i.e.,
impact), very recent innovations cannot be mapped either.

One final aspect we noticed, is that medical imaging research seems to ``stay in
the family'' - so to speak. One imaging modality builds on top of another
modality's knowledge, but the technology hardly seems to diffuse into other
fields. One possible explanation is that biomedical researchers have plenty of
very specific medicine-related journals to choose from for publishing, so they
don't branch out. They do not interact much with journals that have similar
content, but focus on another sector. From personal experience, this is very
apparent in the image processing literature. For example, one can browse image
processing journals from IEEE in vain without finding a solution for a generic
non-medical problem. Meanwhile, medical researchers might have come up with such
a solution years ago, applied it to medicine and published it in a medical image
processing journal.