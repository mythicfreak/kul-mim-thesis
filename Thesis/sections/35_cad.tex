\section{CAD invention: machine learning techniques for mammography}
For this last assessment, we draw from the field of Computer-aided Diagnosis.
For the rest of this section, we will focus on the usage of machine learning
techniques for CADe in mammography. Breast cancer is one of the deadliest
cancers among women today, but fortunately early detection significantly
improves the chances of survival \cite{mammoairecent}. To detect breast cancer,
phycisians look for calcifications, masses and architectural distortions on
high resolution radiographs (mammograms). CAD can help them locate each of these
abnormalities.

Note that - unlike tangible inventions - software algorithms are not so
straightforward to assess using the radical innovation framework. For example,
algorithms typically do not exploit natural effects directly (but computers do).
However, we will make an effort to make a meaningful assessment regardless.

%\cite{mammoai}

\subsection{Defining the technology}
As explained in \autoref{ssec:cadadv}, machine learning comprises a large group
of methods and techniques. We are specifically interested in (binary)
classifiers to determine whether a particular structure on a mammogram is
suspicious enough. The exact nature of this classifier - whether it is a Support
Vector Machine (SVM) or Random Forests (RF) or anything else - is of little
interest for this assessment. We will simply look at them as one group with one goal.

When assessing this technology, we will compare it to earlier incarnations of
CAD software without machine learning elements, and - where appropriate - with
manual diagnosis by a physician.

\subsection{Assessing novelty in functionality}
Because we are dealing with software here, we can only assess the novel
functionality based on novelty in components, not on novelty in natural effects
exploited. 

\subsubsection{Novelty in components}
In \autoref{ssec:cadtech}, we discussed the generic scheme that most CAD systems
follow. Each of these steps can be considered a separate component in the
algorithm. In our case, the old step III and IV components are replaced by
machine learning algorithms.

Of course these algorithms were used before in a variety of other applications,
but not necessarily related to diagnostic medical imaging
\cite{machinelearningapps}. 

\subsubsection{Scores}
In \autoref{tbl:funcscores5} we give an overview of the scores on the various
topics.

\begin{table}[h]
\centering
\begin{tabular}{l l}
\hline
\multicolumn{2}{|c|}{Novelty in functionality} \\
\hline
A. Novelty of components & B. Novelty in natural effects exploited\\
A1) 5 & B1) N/A\\ 
A2) 4 & B2) N/A\\ 
\hline
\end{tabular}
\caption{Novelty in functionality scores}
\label{tbl:funcscores5}
\end{table}

\subsection{Assessing novelty in knowledge origins}
As usual, we start by listing problems encountered during development of this
invention, and present the proposed solution and its related knowledge origin.

Problem 1: Detect abnormalities in mammographs

Problem 1.1: Segment the region of interest

Solution 1.1: Trivial, the mammograph only contains the region of interest

Problem 1.2: Enhance abnormalities

Solution 1.2: Use traditional image processing and computer vision methods

KO1: signal processing, image processing, computer vision 

Problem 1.3: Detect and segment abnormalities

Solution 1.3: TODO

KO2: statistics, artificial intelligence, machine learning

Problem 1.4: Perform feature analysis and classification

Solution 1.4 TODO

KO2: statistics, artificial intelligence, machine learning

Problem 1.5: Reduce false positives

Solution 1.5: Use traditional false positive reduction methods

KO3: 

\subsubsection{Scores}
\begin{table}[h]
\centering
\begin{tabular}{l l}
\hline
\multicolumn{2}{|c|}{Novelty in knowledge origins} \\
\hline
A. Novelty of scientific origins & B. Novelty of technological origins\\
A1) & B1)\\ 
A2) & B2)\\ 
\hline
\end{tabular}
\caption{Novelty in knowledge origins scores}
\label{tbl:origscores5}
\end{table}

\subsection{Assessing technological impact}
\subsubsection{Performance increase}
Already in 1990, \cite{cadsynergy} proved using observer studies that
radiologists' performance in detecting microcalcifications could increase when
using CAD systems, even if the number of false positives at the time were still
fairly high. The review article \cite{cadhistory} looked into various large
scale studies regarding CAD in mammography, and found that all of them reported
an increase in detection performance compared to pre-CAD diagnosis. Remember
that performance is this context is the combined performance of physician plus
computer, not computer alone. It should be mentioned that other studies found no
performance gain, or even a performance decrease \cite{mammocadbad}, but they
seem to be in the minority.

\subsubsection{Technological accumulation}
\paragraph{Broadness of impact}

\paragraph{Magnitude of impact}

\paragraph{Novelty of impact}

\subsubsection{Obsoleting previous technologies}
Within three years of FDA approval, about 10\% of of U.S. facilities switched to
CAD technology\footnote{\url{http://www.nih.gov/news/pr/apr2007/nci-04b.htm}}.
CAD technology using machine learning has not yet obsoleted conventional
mammography diagnosis, although performance definitely increases when employed.
Some physicians are simply reluctant to rely on technology for performing their
diagnosis. This will require a change in mindset, which simply takes time.
On top of that, these systems imply an additional cost on top of the already
expensive imaging equipment. Perhaps integrating them as modules in PACS as
suggested by \cite{cadhistory} will speed up their adoption. For these reasons,
researchers believe it is only a matter of time before such methods are used
globally.

\subsubsection{Scores}
\begin{table}[h]
\centering
\begin{tabular}{l l l}
\hline
\multicolumn{3}{|c|}{Technological impact} \\
\hline
A. Performance increase & B. Tech. accumulation & Obsoleting previous tech.\\
A) 7  & B1 a)  --- b)  & C) 4\\ 
     & B2 a)  --- b)  & \\
     & B3 a)  --- b)  & \\
\hline
\end{tabular}
\caption{Technological impact scores}
\label{tbl:impactscores5}
\end{table}