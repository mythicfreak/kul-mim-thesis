\section{CAD invention: }
For this last assessment, we draw from the field of Computer-aided Diagnosis
(CAD). This is one of the major research areas in diagnostic medical imaging
\cite{cadhistory}. Instead of focusing on the image creation process, CAD
technologies help radiologists with the actual diagnosis of the image. You can
think of CAD as a second opinion. This is in contrast with Automated Computer
Diagnosis (ACD) where the computer does all the work and (in theory) no
physician is needed anymore.

A (non-exhaustive) list of possible application of CAD include detection of
vertebral fractures on radiographs, detection of microcalcifications on
mammographs, detection of cranial aneurysms in MRA images and detection of lung
nodules in CT scans \cite{cadhistory}.

US4907156 \cite{chan1990method}

Early in the 1980's, when research on this topic began, the most common
techniques fell under basic image processing. For example, a special filter was
invented to accentuate lung nodules, and another one to suppress them. These
filters were then both applied to the same image, and the suppressed
was subtracted from the accentuated version. The result should then be a
completely black image, save for the nodule-like structures in the image.

(NASA img procession)

AI, machine learning

\subsection{Defining the technology}

\subsection{Assessing novelty in functionality}

\subsubsection{Novelty in components}

\subsubsection{Novelty in natural effects exploited}

\subsubsection{Scores}
In \autoref{tbl:funcscores5} we give an overview of the scores on the various
topics.

\begin{table}[h]
\centering
\begin{tabular}{l l}
\hline
\multicolumn{2}{|c|}{Novelty in functionality} \\
\hline
A. Novelty of components & B. Novelty in natural effects exploited\\
A1) & B1)\\ 
A2) & B2)\\ 
\hline
\end{tabular}
\caption{Novelty in functionality scores}
\label{tbl:funcscores5}
\end{table}

\subsection{Assessing novelty in knowledge origins}
\subsubsection{Novelty of scientific origins}

\subsubsection{Novelty of technological origins}

\subsubsection{Scores}
\begin{table}[h]
\centering
\begin{tabular}{l l}
\hline
\multicolumn{2}{|c|}{Novelty in knowledge origins} \\
\hline
A. Novelty of scientific origins & B. Novelty of technological origins\\
A1) & B1)\\ 
A2) & B2)\\ 
\hline
\end{tabular}
\caption{Novelty in knowledge origins scores}
\label{tbl:origscores5}
\end{table}

\subsection{Assessing technological impact}
\subsubsection{Performance increase}

\subsubsection{Technological accumulation}
\paragraph{Broadness of impact}

\paragraph{Magnitude of impact}

\paragraph{Novelty of impact}

\subsubsection{Obsoleting previous technologies}

\subsubsection{Scores}
\begin{table}[h]
\centering
\begin{tabular}{l l l}
\hline
\multicolumn{3}{|c|}{Technological impact} \\
\hline
A. Performance increase & B. Tech. accumulation & Obsoleting previous tech.\\
A)   & B1 a)  --- b)  & C) \\ 
     & B2 a)  --- b)  & \\
     & B3 a)  --- b)  & \\
\hline
\end{tabular}
\caption{Technological impact scores}
\label{tbl:impactscores5}
\end{table}