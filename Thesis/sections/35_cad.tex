\section{CAD invention: machine learning techniques for mammography}
For this last assessment, we draw from the field of Computer-aided Detection and
Diagnosis. For the rest of this section, we will focus on the usage of machine
learning techniques for CADe in mammography. Breast cancer is one of the
deadliest cancers among women today, but fortunately early detection
significantly improves the chances of survival \cite{mammoairecent}. To detect
breast cancer, phycisians look for calcifications, masses and architectural
distortions on high resolution radiographs (mammograms). Traditionally, every
image is checked by at least two radiologists to increase sensitivity. This is
known as the second reader principle. However, this approach effectively doubles
the workload of the radiology department. Perhaps more than in any other
medical field, CAD can help radiologists by acting as a surrogate second reader
for mammograms. \cite{mlinmedical}.

One particular application where CAD has proven its worth, is in the detection
of microcalcifications in mammograms. These are small calcium deposits of 0.05mm
to 1mm in size that appear as bright white spots on the scan. They are known to
appear in 30-50\% of all breast cancer cases, and are thus an important
indicator. Due to their variable shape, brightness and size, they can be
difficult to detect in the surrounding tissue \cite{mlinmedical}.

Note that - unlike tangible inventions - software algorithms are not so
straightforward to assess using the radical innovation framework. For example,
algorithms typically do not exploit natural effects directly (but computers do).
However, we will make an effort to make a meaningful assessment regardless.

%\cite{mammoai}

\subsection{Defining the technology}
As explained in \autoref{ssec:cadadv}, machine learning comprises a large group
of methods and techniques. We are specifically interested in (binary)
classifiers to determine whether a particular structure on a mammogram is
suspicious enough. The exact nature of this classifier - whether it is a Support
Vector Machine (SVM) or Random Forests (RF) or anything else - is of little
interest for this assessment. We will simply look at them as one group with one
goal.

When assessing this technology, we will compare it to earlier incarnations of
CAD software without machine learning elements, and - where appropriate - with
manual diagnosis by a physician.

\subsection{Assessing novelty in functionality}
Because we are dealing with software algorithms, we can only assess the novel
functionality based on novelty in components, not on novelty in natural effects
exploited. 

\subsubsection{Novelty in components}
In \autoref{ssec:cadtech}, we discussed the generic scheme that most CAD systems
follow. Each of these steps can be considered a separate component in the
algorithm. In our case, step III and IV are replaced by machine learning
algorithms.

Of course these algorithms were used before in a variety of other applications,
but not necessarily related to diagnostic medical imaging
\cite{machinelearningapps}. 

\subsubsection{Scores}
In \autoref{tbl:funcscores5} we give an overview of the scores on the various
topics.

\begin{table}[h]
\centering
\begin{tabular}{l l}
\hline
\multicolumn{2}{|c|}{Novelty in functionality} \\
\hline
A. Novelty of components & B. Novelty in natural effects exploited\\
A1) 5 & B1) N/A\\ 
A2) 4 & B2) N/A\\ 
\hline
\end{tabular}
\caption{Novelty in functionality scores}
\label{tbl:funcscores5}
\end{table}

\subsection{Assessing novelty in knowledge origins}
As usual, we start by listing problems encountered during development of this
invention, and present the proposed solution and its related knowledge origin.
We again follow the five step detection scheme introduced before. 

Problem 1: detect abnormalities in mammograms

Problem 1.1: segment the region of interest

Solution 1.1: trivial, the mammogram only contains the region of interest

Problem 1.2: enhance abnormalities

Solution 1.2: use traditional image processing and computer vision methods
(e.g.\ convolution filters)

KO1: image processing

Problem 1.3: detect and segment abnormalities

Problem 1.3.1: detect abnormalities

Solution 1.3.1: use an appropriately trained machine learning classifier

KO2: statistics, artificial intelligence, machine learning

Problem 1.3.2: segment abnormalities

Solution 1.3.2: use traditional image processing and computer vision methods
(e.g.\ region growing)

KO3: image processing

Problem 1.4: perform feature analysis and classification

Solution 1.4: use an appropriately trained machine learning classifier

KO4: statistics, artificial intelligence, machine learning

Problem 1.5: reduce false positives

Solution 1.5: use traditional false positive reduction methods

KO5: statistics, machine learning

\subsubsection{Scores}
Of the listed knowledge origins, we classify statistics as scientific, and
the rest as technological.

Statistics forms the fundamental basis for almost all machine learning
algorithms. Some more primitive image processing techniques also explicitly use
statistical theory, but most do not. The radiologists that perform a manual
diagnosis use their advanced human visual system instead of relying on
statistics.

Of the technological knowledge origins, machine learning is by definition the
only new element compared to earlier methods.

\begin{table}[h]
\centering
\begin{tabular}{l l}
\hline
\multicolumn{2}{|c|}{Novelty in knowledge origins} \\
\hline
A. Novelty of scientific origins & B. Novelty of technological origins\\
A1) 6 & B1) 3\\ 
A2) 1 & B2) 5\\ 
\hline
\end{tabular}
\caption{Novelty in knowledge origins scores}
\label{tbl:origscores5}
\end{table}

\subsection{Assessing technological impact}
\subsubsection{Performance increase}
Already in 1990, \cite{cadsynergy} proved using observer studies that
radiologists' performance in detecting microcalcifications could increase when
using CAD systems, even if the number of false positives at the time were still
fairly high. The review article \cite{cadhistory} looked into various large
scale studies regarding CAD in mammography, and found that all of them reported
an increase in detection performance compared to pre-CAD diagnosis. Remember
that performance is this context is the combined performance of physician plus
computer, not computer alone. It should be mentioned that other studies found no
performance gain, or even a performance decrease \cite{mammocadbad}, but they
seem to be in the minority. They particularly lament the high number of false
positives, claiming that these cause more unnecessary examinations and
consequently an increase in medical insurance expenditures.

\subsubsection{Technological accumulation}
false positive reduction
\paragraph{Broadness of impact}
Machine learning is generally application-agnostic and thus a very versatile
technique used in a variety of fields, from the financial world over social
networks to medical applications. In fact, CAD was fairly late to jump on the
machine learning bandwagon \cite{mlinmedical}. Nonetheless, a lot of related
research is performed by biomedical scientists. This research is often generic
enough in nature to potentially be applied to other fields again. Unfortunately,
researchers outside of the medical field tend to ignore medical journals in
favor of their own field-specific alternatives. This severely degrades the
possible cross-pollination across fields, and in turn the broadness of impact.
This is evident by the lack of medical literature citations from outside the
field.

\paragraph{Magnitude of impact}
Globally speaking, the medical field only accounts for a relatively small slice
of all machine learning research. Consequently, most breakthroughs will
originate elsewhere, negatively impacting the magnitude of direct and
indirect impact of CAD on unrelated applications.

\paragraph{Novelty of impact}
Due to the versatility of machine learning, there is a lot of potential
for novelty of impact. But again, because of invisible walls surrounding the
medical field, we could not locate specific applications that drew from
biomedical machine learning research.

\subsubsection{Obsoleting previous technologies}
Within three years of FDA approval, about 10\% of of U.S. facilities switched to
CAD technology\footnote{\url{http://www.nih.gov/news/pr/apr2007/nci-04b.htm}}.
CAD technology using machine learning has not yet obsoleted conventional
mammography diagnosis, although performance definitely increases when employed.
Some physicians are simply reluctant to rely on technology for performing their
diagnosis. This will require a change in mindset, which is a long-term endeavor.
On top of that, these systems imply an additional cost on top of the already
expensive imaging equipment. Perhaps integrating them as modules in PACS as
suggested by \cite{cadhistory} will speed up their adoption. Because of these
reasons, researchers believe it is only a matter of time before such methods are
used globally.

\subsubsection{Scores}
\begin{table}[h]
\centering
\begin{tabular}{l l l}
\hline
\multicolumn{3}{|c|}{Technological impact} \\
\hline
A. Performance increase & B. Tech. accumulation & Obsoleting previous tech.\\
A) 7 & B1 a) 3 --- b) 5 & C) 4\\ 
     & B2 a) 3 --- b) 2 & \\
     & B3 a) 3 --- b) 2 & \\
\hline
\end{tabular}
\caption{Technological impact scores}
\label{tbl:impactscores5}
\end{table}