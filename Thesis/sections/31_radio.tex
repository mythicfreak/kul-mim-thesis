\section{Radiography invention: digital radiography}
A quick Google search on ``radiography breakthrough" suffices to show that
digital radiography is the most significant invention for basic radiography in
recent decades. As mentioned earlier in \autoref{ssec:recentradio}, digital
radiographs are much easier to store, copy, post-process share compared to their
analogue siblings. Additionally, unlike radiographic film there is no potential
for over- or underexposure. Instead, the output can be rescaled as needed during
post-processing. The ubiquity of digital scanners these days prove that the
advantages outclass the disadvantages. However, these disadvantages do exist.
Analogue images have a very high inherent resolution, and by examining them on
a lightbox the contrast is unmatched by any kind of computer screen. Digital
radiographs do not have necessarily to outclass their analogue counterparts on
these fronts, but they have to achieve a minimum level to make sure
their diagnostic value is not impaired.

\subsection{Defining the technology}
What makes radiography analogue or digital depends on kind of detector used.
Other parts of the scanner such as the X-ray source do not have to be altered
to make the transition. Furthermore, it is not one specific technology that
makes this transition possible. Various components are needed, and for each
of them there are some alternatives as well. On top of that, evolution in other
fields such as computer machinery had to be advanced enough to take full
advantage of the possibilities. 

The first incarnation of digital radiography used storage phosphor to
temporarily store the image information, and lasers to read out the values pixel
by pixel at a later stage \cite{digitalradio}. Unfortunately, the physical
properties of storage phosphors severely limited the resolution of the resulting
image, reducing their diagnostic value. Instead, flat panel detectors are more common
now.



\subsection{Assessing novelty in functionality}

\subsection{Assessing novelty in origins}

\subsection{Assessing technological impact}