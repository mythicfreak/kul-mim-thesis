\chapter{Introduction to diagnostic medical imaging}

\section{Introduction}
In this chapter, we provide an overview of diagnostic medical imaging and its
history. Medical imaging is a field in medicine concerned with creating visual
representations of a body for the purpose of clinical analysis. Although medical
imaging is sometimes used for non-diagnostic purposes, we will only concern
ourselves with diagnostic medical imaging. This subbranch has the goal to
facilitate diagnosis of medical conditions without the need for invasive
procedures.

In the next sections, we will discuss the history of the field, followed by an
introduction to the most important modalities such as CT and MRI scanners.

\section{History of radiography and X-rays}
Most of the topics covered in this chapter build on the work of Wilhelm Konrad
R\"ontgen, a German physicist who produced and detected X-rays for the first
time on November 8 1895. These X-rays (X for unknown) had the remarkable
property of being attenuated at different rates when passing through various
materials. For example, bone strongly attenuates the X-rays while soft tissue
does much less so. R\"ontgen also discovered that the radiation can be captured
on a photographic plate, just like regular light. He presented his findings in
his paper ``On a new kind or rays'' \cite{rontgen}. This discovery earned him
the Nobel Prize in Physics in 1901.

Only two weeks after his discovery, he produced the first X-ray photo of his
wife's hand, showing the inner skeleton. Just a couple of months later, X-rays
were already being used in a clinical setting on patients.

\section{Physical background}
To better understand the internal workings of imaging devices, we present a
simplified mathematical and physical background based on the book of prof.
Suetens\cite{suetens}. X-rays are simply a form of electromagnetic waves
consisting of photons with a wavelength $\lambda$ on the order of Angstr\o ms
($10^{-10}$m). The corresponding frequency $f$ places these rays firmly in the
ionizing (radioactive) part of the spectrum. The energy of such a wave can be
calculated with the following formula, where $c$ is the speed of light and $h$
is Planck's constant.

\begin{equation}
	E = hf = \frac{hc}{\lambda}.
\end{equation}

X-rays are generated in an X-ray tube, a vacuum tube consisting of a cathode and
an anode Current flowing through the cathode releases electrons,
which are accelerated toward the anode by an applied voltage. Once the electrons
hit the anode, they release part of their energy in the form of X-ray photons.

The attenuation of X-rays through materials can easily be modeled using an
attenuation coefficient $\mu$. The beam intensity when passed through a
homogeneous material of depth $d$ is given by: 

\begin{equation}
	I_{out} = I_{in} e^{-\mu d}.
\end{equation}

To capture X-rays, a detector is needed. Traditionally, a screen-film detector
was used. (However, recent advancements allow more performant and completely
digital detectors.) The familiar photographic film alone is very inefficient
at capturing X-rays: only about 2\% of all photons are absorped. Because X-rays
are radioactive, the applied dose cannot simply be increased to improve the
image quality. Instead, an intensifier screen is used in front of the film. This
screen contains heavy chemical elements, whose electrons are excited by the
incoming photons. When returning to their original state, these electrons emit
more photons that can be captured by the film, raising the absorption efficiency
to about 50\%.


\section{X-ray computed tomography}

\section{Magnetic resonance imaging}

\section{Nuclear medicine imaging}

%ultrasound, img processing, automated diagnosis
