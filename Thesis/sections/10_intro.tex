\chapter{Introduction}\label{chap:intro}
\section{Identification of important contributions in a technological field or industry}
In the search for value creation on both the micro- and the macroeconomic scale,
technological innovation has long been considered a very important factor. Some
would argue that it is a key factor in long term growth \cite{arts}. This
innovation comes in many forms, from a simple evolution of the previous state of
the art to a veritable revolution that causes a paradigm shift. To that end, a
lot of research has already been done to clarify what such a revolution or
radical innovation is, and how we can quantify it \cite{structure, invention,
verhoeven}. With this information, researchers hope to one day predict with
high accuracy how much of a breakthrough an idea will be, even before it enters
the market.

\subsection{Innovation framework}
In this text, we will work with the framework proposed by \cite{verhoeven}.
This article proposes both technological and economical dimensions on which to
rank innovation. We will focus on the three technological dimensions: new
knowledge origins, new functionality and impact. Each of these will be explained
in the following paragraphs.

\subsubsection{Novelty in knowledge origins}
The first two dimensions look at the underlying technology of an invention. In
particular, the first dimension is concerned with the knowledge origins. For an
innovation to work as intended, a number of problems usually have to be solved
first. The various sources of knowledge used to solve these problems are
appropriately called knowledge origins. Non-disruptive innovations typically
draw from the same knowledge origins compared to related technologies. On the
other hand, radical innovations often use knowledge from an entirely different
origin. Furthermore, we can distinguish scientific from technological origins.

\subsubsection{Novelty in functionality}
Besides knowledge origins, novelty in functionality is another dimension based
on the underlying technology. That is, radical inventions often use 
new (combinations of) components and principles compared to previous related
innovations.

\subsubsection{Impact}
The third dimension proposed is impact on future technological development.
While technology mostly evolves along a relatively straightforward path,
disruptive innovations often offer a whole new way of thinking - a paradigm
shift. If successful, future innovations are likely to continue down that new
path. In other words, the more the innovation under scrutiny impacts future
developments, the more likely it is to be a radical innovation. Impact can be
direct or indirect, and can be very broad (i.e., affecting other fields than
its own).

\subsubsection{Assessment methodology}
To perform a manual assessment of a certain innovation, some standard questions
related to these three dimensions need to be answered with a score. Those
scores should be substantiated by the relevant literature  or - if possible - by
an expert in the field. The complete assessment sheet can be found in
\autoref{app:score}.

\subsection{Patent indicators}
The framework not only defines the relevant dimensions and manual assessment
methods, but also proposes similar patent indicators. Patents provide a rich
source of data because they not only refer to the inventing entity, but also
offer plenty of forward and backward citations that can be analysed. These allow
us to find the true origin of a new technique, and estimate the impact a patent
had on future research. On top of that, advanced computer algorithms can use
them as a source for data mining and big data techniques. This allows us to
automate part of the process, and gain new insights at the same time. One
limitation of the patent database we work with, is that it only contains
reliable data from 1980 onwards. Because of this, we will constrain ourselves to
assessing inventions in that specific time span.

\section{Goal of this thesis}
In this thesis we will perform a manual assessment of a certain technological
field based on the framework explained above. The field of choice is diagnostic
medical imaging. This was chosen because it sits nicely on the intersection
between biomedical engineering and computer science, two domains the author is
fairly familiar with.

Before we can dive into that, \autoref{chap:imaging} introduces us to
diagnostic medical imaging. Various imaging modalities, such as CT and MRI
scanners will be discussed at length. Once we have a basic understanding of the
most important techniques, \autoref{chap:inventions} will guide us through the
actual assessment of some interesting breakthroughs in the field.

Future research will compare the results of this thesis with the outcome of an
automated assessment based on patent indicators. This will provide the required
feedback for and validation of the proposed framework and patent indicators.

% \subsection{Structure of this text}
% In this chapter, we properly defined our framework of radical innovation.
% Next, \autoref{chap:imaging} introduces us to diagnostic medical imaging, our
% technological field of choice. Various imaging modalities, such as CT and MRI
% scanners will be discussed at length. As soon as we have a basic understanding
% of this field, \autoref{chap:inventions} will highlight some of the most
% interesting breakthroughs since 1980. These breakthrough will be analysed using
% the aforementioned framework. Finally, \autoref{chap:conclusions} concludes this thesis.
