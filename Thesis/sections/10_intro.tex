\chapter{Introduction}
\section{Identification of important contributions in a technological field or industry}
In the search for value creation on the level of firms, technological innovation
has long been considered a very important factor. This innovation comes in many
forms, from a simple evolution of the previous state of the art to a
veritable revolution that causes a paradigm shift. To that end, a lot of
research has already been done to clarify what such a revolution or radical
innovation is, and how we can quantify it. \cite{structure, invention,
verhoeven} With this information, research hope to one day predict how much of
a breakthrough an idea will be, even before it enters the market.

In this text, we will work with the framework suggested by \cite{verhoeven}.
This article proposes both technological and economical dimensions on which to
rank innovation. We will focus on the three technological dimensions: new
knowledge origins, new functionality and impact. Each of these will be explained
in the following paragraphs.

\subsection{Novelty in origins}

\subsection{Novelty in functionality}

\subsection{Impact}

A complete score chart can be found in \autoref{app:score}.

\section{Next chapters}
Now that we have defined our framework of radical innovation,
\autoref{chap:imaging} will provide us with an introduction to diagnostic
medical imaging, our technological field of choice. As soon as we have a basic
understanding of this field, \autoref{chap:inventions} will highlight some of
the most interesting breakthroughs since 1980. These breakthrough will be
analysed using the framework we just defined in this chapter. Finally,
\autoref{chap:conclusions} concludes this thesis.

%TODO link with imaging
%TODO patents
%TODO timespan