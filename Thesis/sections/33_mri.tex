\section{MRI invention}
Normally we only consider inventions from 1980 onwards, but for this section we
would like to make an exception. The invention of the MRI scanner itself was a
veritable breakthrough, and still happened relatively close to 1980. Because we
already had a close look at MRI in \autoref{sec:mri}, we will skip the
technology definition and go straight to the assessment.

\subsection{Assessing novelty in functionality}
\subsubsection{Novelty in components}
The purpose of MRI scanners is largely the same as that of CT scanners: create
cross-sectional images to look into patient's bodies without using invasive
techniques. The essential components of CT are the X-ray source and the
detector. With MRI, we again have a signal source and a signal detector.
However, this time the source is a large electromagnet, and the detector is a
quadrature detector.

Electromagnets are used in various other technologies to do the same thing:
generate magnetic fields. Examples include motors, generators, transformers,
relays and loudspeakers. In that sense the choice was not very special.

Similarly, quadrature detectors are used universally whenever signal
demodulation is required \cite{quadrature}. Remember that MRI operates in the
radio frequency range, so ordinary radio technology could be repurposed for use
in these scanners.

In summary, these components were not used in medical imaging before the advent
of MRI. Yet, once it was decided to exploit radio waves and magnetic fields, the
choice of components was rather trivial.

\subsubsection{Novelty in natural effects exploited}
MRI has one thing in common with the other imaging modalities discussed: they
all use electromagnetic radiation in one form or another. Nevertheless, the way
this natural effect is exploited here is considerably different from other
modalities. In radiography and CT, we measure the X-ray attenuation when
traveling through the human body. With nuclear medicine, we essentially detect
photons emitted by radioactive materials. MRI on the other hand influences and
measures the magnetic properties of tissues, in particular the proton density,
the $T_1$ time or the $T_2$ time.

In a way, every researcher that ever worked on imaging based on magnetic fields
contributed to the eventual creating of MRI. Of course some groups took a
slightly different turn (e.g. measuring properties of electrons instead of
protons), but their larger goal was the same. To our knowledge no one
experimented with these specific magnetic properties for any other purposes.

\subsubsection{Scores}
In \autoref{tbl:funcscores3} we give an overview of the scores on the various
topics.

\begin{table}[h]
\centering
\begin{tabular}{l l}
\hline
\multicolumn{2}{|c|}{Novelty in functionality} \\
\hline
A. Novelty of components & B. Novelty in natural effects exploited\\
A1) 10 & B1) 8\\ 
A2) 3 & B2) 8\\ 
\hline
\end{tabular}
\caption{Novelty in functionality scores}
\label{tbl:funcscores3}
\end{table}

\subsection{Assessing novelty in knowledge origins}
Once again, we start with a list of problems and corresponding solutions.

Problem 1: measure proton density in tissue

Problem 1.1: align net magnetization vectors

Solution 1.1: generate a large magnetic field

KO1: electromagnetism

Problem 1.1.1: increase field strength, lower power consumption

Solution 1.1.1: superconductivity by extreme cooling

KO2: physics (cryogenics), material science

Problem 1.2: measure magnetization vector

Solution 1.2a: disturb equilibrium with the appropriate pulse sequence

KO3: theory from electromagnetism, but to our knowledge never exploited before

Solution 1.2b: record relaxation phenomena using quadrature detector

KO4: analogue electronics, radio equipment 

Problem 2: localize position using signal

Solution 2: apply a magnetic field gradient

KO5: electromagnetism, gradients recently used in \cite{nanogradients1,
nanogradients2}, older: \cite{oldgradients}.

\subsubsection{Scores}
None of these knowledge origins, except the usage of specific pulses, is out of
the ordinary. Radio equipment is also a new technological origin, but its
purpose in other inventions is very similar. That is why in
\autoref{tbl:origscores3} we give low scores except for this technological
origin.

\begin{table}[h]
\centering
\begin{tabular}{l l}
\hline
\multicolumn{2}{|c|}{Novelty in knowledge origins} \\
\hline
A. Novelty of scientific origins & B. Novelty of technological origins\\
A1) 2 & B1) 6\\ 
A2) 2 & B2) 7\\ 
\hline
\end{tabular}
\caption{Novelty in knowledge origins scores}
\label{tbl:origscores3}
\end{table}

\subsection{Assessing technological impact}
\subsubsection{Performance increase}
Although we have portrayed MRI as the successor to CT in the section above, this
is not completely true. MRI allows us to visualize things we could not before
with CT. For example, cancerous tissue is sometimes easier to spot on MRI scans.
Whereas on CT scans the attenuation properties of both tissue types might be
very similar, malignant tumors appear to exhibit a larger $T_1$ value than
surrounding tissue \cite{mrihistory}.

On the other hand, CT has the advantage when it comes to visualizing bone.
Because it contains few free hydrogen atoms, it is very difficult to spot on an
MRI scan. X-rays however are ideal for this type of diagnosis.

In conclusion, it is very difficult to unilaterally award the performance
trophy to either side. 

\subsubsection{Technological accumulation}
\paragraph{Broadness of impact}
MRI generated a lot of spin-off technologies, including Magnetic Resonance
Angiography (MRA), Magnetic Resonance Spectroscopy (MRS) and functional MRI
(fMRI). These are of course all part of the same imaging family, and we could
not find any impact outside of this family in the literature. Perhaps
mining through patents can show some interesting results. 

\paragraph{Magnitude of impact}
Although the broadness is limited, the magnitude of impact is substantial.
Especially for the direct spin-offs listed above, but also indirectly through
the spin-offs of those spin-offs.

\paragraph{Novelty of impact}
As stated above, we only identified direct impact on its immediate spin-off
technologies. The degree of novelty is thus very low.

\subsubsection{Obsoleting previous technologies}
As stated before in \autoref{ssec:mrifuture}, MRI has not replaced CT, but is
expected to gain more traction in the future. For now, the two technologies are
complimentary: each is better for a specific subcategory of diagnostic imaging. 

\subsubsection{Scores}
\autoref{tbl:impactscores3} shows the final scores.

\begin{table}[h]
\centering
\begin{tabular}{l l l}
\hline
\multicolumn{3}{|c|}{Technological impact} \\
\hline
A. Performance increase & B. Tech. accumulation & Obsoleting previous tech.\\
A) 5 & B1 a) 4 --- b) 2 & C) 3\\ 
     & B2 a) 8 --- b) 8 & \\
     & B3 a) 1 --- b) 1 & \\
\hline
\end{tabular}
\caption{Technological impact scores}
\label{tbl:impactscores3}
\end{table}